\documentclass[a4paper,11pt]{report}

\usepackage{caption}
\usepackage{courier}
\usepackage{url}
\usepackage{graphicx}
\usepackage{amssymb,amstext,amsmath}
\usepackage{pstricks}
\usepackage{lscape}
\usepackage{rotating}
\usepackage{pst-node}
\usepackage{listings}
\usepackage{makeidx}
\usepackage{pst-blur}
\usepackage{geometry}
\usepackage{times} 
\usepackage{setspace}
\usepackage{algorithmic}
\usepackage{algorithm}
\DeclareCaptionFont{white}{\color{white}}
\DeclareCaptionFormat{listing}{\colorbox[cmyk]{0.43, 0.35, 0.35,0.01}{\parbox{\textwidth}{\hspace{15pt}#1#2#3}}}
\captionsetup[lstlisting]{format=listing,labelfont=white,textfont=white, singlelinecheck=false, margin=0pt, font={bf,footnotesize}}
%
\begin{document}
%----------------------------------------------------------
%\renewcommand{\chaptermark}[1]{\markboth{\MakeUppercase{#1}}{}}
%\renewcommand\chaptername{Test}
%----------------------------------------------------------
\lstset{
         basicstyle=\footnotesize\ttfamily, % Standardschrift
         %numbers=left,               % Ort der Zeilennummern
         numberstyle=\tiny,          % Stil der Zeilennummern
         %stepnumber=2,               % Abstand zwischen den Zeilennummern
         numbersep=5pt,              % Abstand der Nummern zum Text
         tabsize=2,                  % Groesse von Tabs
         extendedchars=true,         %
         breaklines=true,            % Zeilen werden Umgebrochen
         keywordstyle=\color{red},
            frame=b,         
 %        keywordstyle=[1]\textbf,    % Stil der Keywords
 %        keywordstyle=[2]\textbf,    %
 %        keywordstyle=[3]\textbf,    %
 %        keywordstyle=[4]\textbf,   \sqrt{\sqrt{}} %
         stringstyle=\color{white}\ttfamily, % Farbe der String
         showspaces=false,           % Leerzeichen anzeigen ?
         showtabs=false,             % Tabs anzeigen ?
         xleftmargin=17pt,
         framexleftmargin=17pt,
         framexrightmargin=5pt,
         framexbottommargin=4pt,
         %backgroundcolor=\color{lightgray},
         showstringspaces=false      % Leerzeichen in Strings anzeigen ?        
 }
 \lstloadlanguages{% Check Dokumentation for further languages ...
         %[Visual]Basic
         %Pascal
         C
         %C++
         %XML
         %HTML
         %Java
 }
    %\DeclareCaptionFont{blue}{\color{blue}} 


%\lstset{ %
%language=C++,                % choose the language of the code
%basicstyle=\footnotesize,       % the size of the fonts that are used for the code
%numbers=left,                   % where to put the line-numbers
%numberstyle=\footnotesize,      % the size of the fonts that are used for the line-numbers
%stepnumber=5,                   % the step between two line-numbers. If it's 1 each line will be numbered
%numbersep=5pt,                  % how far the line-numbers are from the code
%%backgroundcolor=\color{},  % choose the background color. You must add \usepackage{color}
%showspaces=false,               % show spaces adding particular underscores
%showstringspaces=false,         % underline spaces within strings
%keywordstyle=\color{green},
%showtabs=false,                 % show tabs within strings adding particular underscores
%frame=single,           % adds a frame around the code
%tabsize=2,          % sets default tabsize to 2 spaces
%captionpos=b,           % sets the caption-position to bottom
%breaklines=true,        % sets automatic line breaking
%breakatwhitespace=false,    % sets if automatic breaks should only happen at whitespace
%escapeinside={\%*}{*)}          % if you want to add a comment within your code
%}
%-----------------------------------------------------------
\begin{titlepage}
 
\begin{center}

\ \\[1.5cm] 
\textbf{\Large DEDUPLICATION AND COMPRESSION BENCHMARKING SUPPORT IN FILEBENCH}\\[1.5cm]
 
 \textbf{ \small CSE506 Project Report}\\[1.5cm]

\textbf{By\\Rami Al-Rfou'\\ Nikhil Patwardhan \\ Phanindra Bhagavatula}\\[1.5cm]
{\textit{ \today}} 


\end{center}
\vfill
\end{titlepage}

%-----------------------------------------------------------
\tableofcontents
%-----------------------------------------------------------
\begin{abstract}
Deduplication systems look for repeating patterns of data at the block and bit levels. When multiple instances of the same pattern are discovered, the system stores a single copy of the pattern. However, most of the popular file-system benchmarks generate data in a manner that does not enable realistic benchmarking of deduplication systems. Controlling the entropy of the data that is used while benchmarking deduplication systems is one way of overcoming this limitation. In this project, we integrated entropy-based data generation in the Filebench file-system benchmarking suite.

By assuming the emission of a byte in the data stream as an event, we generate data that can take values from 0 to 8 bits/byte. The user is able to control the value of entropy by specifying it in the workload to be run. We show that by varying the amount of entropy of data that is either written to or read from a disk using a deduplicated file-system, the benchmarking results are more pertinent to the actual behavior of such a file-system.
\end{abstract}

\chapter{Introduction}\label{chap:intro}

\section{Filebench}

testing\cite{Simpson}

\section{SDFS}


\chapter{Design}\label{chap:des}

In our design we tries the following principles:

\begin{itemize}

\item \textbf{Minimal change} %suggest better name%

We tried our best to make the scope of the changes as minimal as possible. This applies to the size of the patch counted by number of lines 
and the number of files modified. 

The files that are modified
\begin{itemize}
\item fileset.c
\item flowop\_library.c
\item parser\_lex.l
\item parser\_gram.y
\end{itemize} 

\item \textbf{Extensibility} \\

\item \textbf{Backward compatibility} \\
The patched Filebench runs all the old workload model files without modification. The patch is triggered only when the data source attribute is specified. The patch is surrounded by conditional compilation preprocessors that enables the user to switch the functionality on or off at the compiling time.

\item \textbf{Modularity}\\
Any code that do not change the flow of the current Filebench code base, is separated and kept in separate \verb+C+ modules.
 Files added
\begin{itemize}
\item sources.c/sources.h 
\item entropy.c/entropy.h
\end{itemize}

\end{itemize}



\section{Filebench Interpreter}
The changes in the Filebench Interpreter have been designed to enable Filebench to optionally accept various datasources. These datasources are meant to be used to populate files created during benchmarking process. The Interpreter has thus been modified to accept new attributes called \textbf{datasource} and \textbf{entropy} as a part of the fileset command. For example a valid fileset command is:\\
\indent \verb+define fileset name=bigfileset,path=\$dir,...,datasource=entro,entropy=3.4+

The attribute \textbf{entropy} is not directly an attribute of \textbf{fileset} command. It is a subattribute of the datasource type. For example the below is an invalid fileset definition.\\
\indent \verb+define fileset name=bigfileset,path=\$dir,...,entropy=3.4+

\noindent All attributes for command \verb+fileset+ have a place in the \verb+struct fileset+. The \verb+datasouce+ attribute has also been placed in the structure.

\noindent A new place has been created in this structure for \textbf{datasource} attribute alone and not for \textbf{entropy} attribute. The logic is that, the only attribute that makes sense to be part of \textbf{fileset} is the type of data. The attributes which define the data itself of the \textbf{datasource} are irrelevant to be part of \textbf{fileset} datastructure.\\
\noindent Since the a datasource can have various attributes of the data constituting it, \verb+datasource+ object has a pointer to a list of attribute objects relevant to it. This is the reason \verb+datasource+ is of type attr and not avd\_t. \\
\noindent To accommodate  a list of sub-attributes inside an attribute(like the \verb+datasource+), \verb+struct attr+ has been modified.




\textbf{extensibility}

The design of the parser has been made such that a new datasouce can be easily defined and any number of sub-attributes of this datasource can be specified without any significant amount of code change.


\noindent 



%Rami:
%Can you please state the old and new grammars
%show snippets from the important code
%describe the behavior in all the cases that can face the interpreter
%datasource specified or not * entropy specified or not

\section{Filebench Workflow Integration}
The integration in Filebench workflow can be logically divided into the following parts:

\subsection{Data Specification}

Data intended to be written to or read from files could be specified in many ways for benchmarking purposes. One approach is by specifying its \textit{entropy}.
 However, in future, users may have more specific requirements like populating a certain character in the data or a specific distribution for the data.
 We currently support only entropy based data population, but keeping possible future requirements in mind we chose to create a separate structure which is dedicated to store all the specifications about the data itself.
One of its fields is a function pointer that is set \textit{dynamically} depending on the type of data requested. Since data specification is made per \textit{fileset}, we put this structure as a member of \verb+struct fileset+.

\subsection{Fileset Initialization}
The \texttt{struct fileset} structure holds information that is relevant to a set of files. Entropy is specified per \textit{fileset} and it is recorded in this structure by the parser, as explained in section \ref{sec:parse_des}. This information is used to appropriately initialize the structure described above. Its function pointer is dynamically pointed to a function that provides data in the desired format. Once initialized, this function pointer can be used by the corresponding \textit{flow operations} to populate their buffers in a generic manner.

\subsection{Flow Operations}
Flow operations represent workload actions and they carry out buffered I/O on the files. Using the dynamically set function pointer, we populate the buffer (in this case with the specified entropy) just before writing it to an open file in the following flow operations:
\begin{enumerate}
\item write
\item writewholefile
\item appendfile
\item appendfilerand
\end{enumerate}

\section{Entropy Generator}\label{sec:ent_des}
\subsection{Entropy function}\label{sub:func}
The entropy is quantity that is defined for a set of data to quantify how much random it is.
For any stream of data that is composed of $n$ symbols, its entropy is given by the following
equation:
\begin{equation}\label{eq:ent}
Entropy = -\sum_{i=1}^n P(s_i)\lg P(s_i)
\end{equation}

$P(s_i)$ is the probability that a symbol $s_i$ is generated by the data generator.
To calculate the probability of a symbol in a generated data, can be done by calculating
the experimental probability according to equation \ref{eq:exp}
\begin{equation}\label{eq:exp}
P(s) = \frac{number\, of\, s\, occurences}{size\, of\, data}
\end{equation}

The maximum entropy that can be obtained from a data generated using $n$ symbols is when the
probability distribution of symbols is uniform. That entropy of a uniform distribution can by calculated as following
\begin{align}
Entropy &= -\sum_{i=1}^n P(s_i)\lg P(s_i) \nonumber \\
        &= -\sum_{i=1}^n \frac{1}{n} \lg \frac{1}{n} \nonumber \\
        &= \frac{-1}{n}\sum_{i=1}^n -\lg n \nonumber \\
        &= \lg n 
\end{align}

To decrease the value of entropy, we can imbalance the uniform distribution. Using this method we can reach zero by increasing the probability of the first symbol to 1 and decreasing the
others to zero. However, calculating the $\epsilon_i$ that should be added to subtracted from every $P(s_i)$
to reach a specific entropy value less than $\lg n$ can be hard. To simplify the situation we 
can focus on changing the probability of two symbols at a time.
Our target to generate all values of $x \ni \lg(n-1)<x<\lg n$ just by changing the probability of two symbols.

\begin{align}
Entropy &= -[(P(s_1)+\epsilon)\lg (P(s_1)+\epsilon) + (P(s_2)-\epsilon)\lg (P(s_2)-\epsilon)\nonumber \\
        &\qquad{} +  \sum_{i=3}^n P(s_i)\lg (P(s_i))]\label{eq:imb_ent}\\
 &= -[(\frac{1}{n}+\epsilon)\lg (\frac{1}{n}+\epsilon) + (\frac{1}{n}-\epsilon)\lg (\frac{1}{n}-\epsilon) \nonumber \\
      &\qquad  +  \frac{n-2}{n}\lg (\frac{1}{n})]\label{eq:imb_ent2}
\end{align}


To prove that the entropy as a function of $\epsilon$ equal to any value between $\lg n, \lg (n-1)$, the maximum entropy using $n$, $n-1$ symbols respectively. We notice the following observations:\\
\begin{itemize}
\item Equation \ref{eq:imb_ent2} shows that entropy is a function in one variable, $\epsilon$.
\item Equation \ref{eq:imb_ent2} also shows that the entropy is a continuous function of $\epsilon$ on the interval $\epsilon \in [0,1/n)$ as it is a result of adding and multiplying continuous functions on the same interval.
\item Entropy is equal $\lg n$ when $\epsilon = 0$.
\item Entropy is equal to $\lg(n) -\frac{2}{n}$  when $\epsilon$ reaches $1/n$
\item $\lg(n) - \frac{2}{n} < \lg(n-1) \,\,\, \forall n > 2$ \footnote{Can be verified by visiting\\ 
\url{http://www.wolframalpha.com/input/?i=lg\%28n-1\%29+-+\%28lg\%28n\%29++-+2\%2Fn\%29&a=*FunClash.lg-_*Log2.Log10-}}.
\end{itemize}
Given the above and using the median value theorem the Entropy function spans over the interval $(\lg (n-1), \lg n)$ using subinterval of $\epsilon$ values.
To get the value of $\epsilon$ we can apply any numerical method to find the roots of the equation.

\subsection{Random generator}\label{sub:gen_des}
In \ref{sub:func} we showed that any entropy value can be obtained using a slightly modified uniform distribution.
Now, given that the probability distribution function (PDF) of our symbols is already calculated.
How can we build a data source that generates the data with the given entropy ?
Again we will use a uniform random source to help us.The idea that we will calculate the cumulative distribution
function (CDF) of the symbols table first. Then use the output of a uniform random generator to search for the
corresponding symbol of the random value. Every symbol has different size interval that correspond to its probability.
Because the CDF is an increasing function, the CDF table is increasing also which allow us to use in our search a binary
search algorithm. 
\begin{algorithm}
\caption{Random Generator}
\label{alg:rnd}
\begin{algorithmic}
\STATE Solve the equation to get the value of $\epsilon$
\STATE Calculate PDF
\STATE Calculate CDF
\FOR{$i = 1$ \TO Buffer size} 
\STATE $x$= Random number in $[0,1)$
\STATE index = search in which interval of CDF $x$ lie.
\RETURN SymbolTable[index]
\ENDFOR
\end{algorithmic}
\end{algorithm}


Because the symbol table has constant size, the cost of the binary search is also constant.
This guarantees that the time complexity of our algorithm is linear. However, in \ref{sec:ent_imp} we will show that in practice the 
constant factors of such algorithm is not good enough. Moreover, will will present other different modifications
of the algorithm.


\section{Design Principles}
During the process of designing we tried our best to maintain the following principles. 

\begin{itemize}

\item \textbf{Minimal change} %suggest better name%

The scope of the changes is as minimal as possible. This applies to the size of the patch counted by number of lines and the number of files modified that were modified. 

The files that are modified
\begin{itemize}
\item fileset.c/fileset.h
\item flowop\_library.c
\item parser\_lex.l
\item parser\_gram.y
\end{itemize}
Moreover, we used any already available structure instead of reinventing the wheel.

\item \textbf{Extensibility} \\

\item \textbf{Backward compatibility} \\
The patched Filebench runs all the old workload model files without modification. The patch is triggered only when the data source attribute is specified.
 The patch is surrounded by conditional compilation preprocessors,\verb+CONFIG_ENTROPY_DATA_EXPERIMENTAL+ , that enables the user to switch the functionality on or off at the compilation time.

\item \textbf{Modularity}\\
Any code that do not change the flow of the current Filebench code base, is separated and kept in separate \verb+C+ modules.
 Files added
\begin{itemize}
\item sources.c/sources.h 
\item entropy.c/entropy.h
\end{itemize}

\end{itemize}

\chapter{Implementation}\label{chap:imp}

\section{Filebench Interpreter}
The implementation of the Interpreter has been done in fashion that future extensibility of Filebench to accept different datasources with optional and variable number of sub-attributes is easy. Most of the code changes have been made to the \verb+parser_gram.y+ and \verb+parser_lex.l+ files. \verb+parser_lex.l+ file has a list of valid tokens. Two tokens (\verb+datasource+ and \verb+entropy+) have been added to this file. Since the parser never recognized decimal numbers e.g. 3.4 earlier, the file has also been modified to accept decimal values. New tokens in the \verb+parser_lex.l+ are as below:

\lstset{language=C}
\begin{lstlisting}
#include "utils.h"
#include "parser_gram.h"
..
%%
..
datasource              { return FSA_DSRC;}
entropy                 { return FSA_ENTROPY;}
..
<INITIAL>[0-9]*\.[0-9]+  {  .. } // parse decimal values.
..
%%
..
\end{lstlisting}

\noindent \verb+parser_gram.y+ file has been modified to accept \verb+datasource+ as a parameter. \verb+entropy+ attribute is accepted by the grammar only if \verb+datasource+ parameter is present.\\ Following are the key rules in the grammar.

\lstset{language=C}
\begin{lstlisting}

..
%%
..
//define fileset command supports only source_type and not entropy

files_define_command: FSC_DEFINE FSE_FILE { .. }
| FSC_DEFINE FSE_FILESET { .. }
| files_define_command files_attr_ops { .. }
| files_define_command files_attr_ops FSK_SEPLST \bf{source_type} { .. }     
..

source_type: FSA_DSRC FSK_ASSIGN FSV_STRING { .. }
| FSA_DSRC FSK_ASSIGN FSV_STRING FSK_SEPLST source_define_params { .. } 	

..

// Support for multiple sub-attributes for datasource
source_define_params: source_define_param { .. }
| source_define_params FSK_SEPLST source_define_param { .. }	        
..

source_define_param: source_params_name FSK_ASSIGN attr_value { .. }

%%
..
\end{lstlisting}


To keep the parser generic, it was decided that the parser won't verify if a sub-attribute (like \verb+entropy+) is valid for a particular type of \verb+datasource+ (like ``entro"). 

The sub-attributes are stored in a list of type \verb+struct attr+ and the \verb+datasource+ object has a pointer to this list. Following is the new variable in \verb+fileset+ structure pointing to the \verb+datasource+ object.

\lstset{language=C}
\begin{lstlisting}
typedef struct fileset { 
struct fileset  *fs_next;  
avd_t       fs_name; 
avd_t       fs_path;   
avd_t       fs_leafdirs; 
..
avd_t       fs_dirwidth;
..
struct attr *fs_datasource;
..
};
\end{lstlisting}







\section{Entropy Generator}\label{sec:ent_imp}

The entropy generator is organized into two modules \verb+entropy.c+ and \verb+source.c+ . \verb+entropy.c+ is a library that contains a group of helping functions to build
 data sources with random generation capabilities. \verb+source.h+ has the definition of the \verb+source+ and \verb+source_operations+ structures.
\lstset{language=C}
\begin{lstlisting}
struct source {
    double s_entropy;
    struct source_operations *s_ops; 
};

struct source_operations {
    int (*fill)(struct source *, void *, unsigned int);
};

\end{lstlisting}


 In \verb+sources.c+ we can find few declared instances of the later structure. Filebench uses \verb+dummy_operations+ by default unless an entropy is specified as a parameter to the fileset declaration statement.
 In that case, Filebench will assign \verb+entropy_operations+ to the fileset source structure. Both structures are minimal and can be expanded for any future needs.

\subsection{Calculating the pdf}

As explained in section \ref{sub:ent_des} we explained the algorithm to calculate the pdf.
However, the equation is not easily solvable analytically. To overcome this problem. Numerical methods can be used to solve the problem. We implemented the secant method to find the roots of the equation. The secant method was chosen as it is easy to implement, converge fast enough, only 20 iterations needed !. Not to mention that the requirements are easy to prepare, you have to specify the range that you expect the solution to be in.


\subsection{Populating algorithm}

Filebench is calling \verb+entropy_fill+ operation to fill the buffer with data with the specific entropy. \verb+entropy_fill+ is an interface that call the actual algorithm that calculate the PDF and populate the data.

Many algorithms were proposed to generate data, differs mainly in the way they generate the data stream out of the calculated PDF.

In section \ref{sub:ent_des} we explained how can we map the PDF to generate symbols according to their probability. This method is described in literature as roulette selection algorithm, our own implementation is called
 \verb+entropy_search_fill+. Although the time complexity of such algorithm is linear. The constant factors can go up to 10; the binary search takes at most 8 comparisons as the symbols table size is 256.


In an effort to minimize the time used to fill the buffer the following methods were implemented:

\begin{itemize}
\item \verb+entropy_cont_fill+. \\
   fills the buffer with random data according to
    the pdf. It will generate contiguous segments of data in the
    buffer to make different same size buffer look different, we
    shuffle the symbols table. It takes less time but it does not
    give homogeneous entropy data stream. Because of rounding errors
    we will have some remaining elements are not filled, so we will
    use the search method to fill them.
\item \verb+entropy_permutate_fill+\\
    fills the buffer with random data according to
    the pdf. It will generate contiguous segments of data in the
    buffer. After that a permutation function will be called to make
    the buffer looks with homogeneous entropy. The permutation
    function has high overhead, because it is not cache friendly.
    tests show 16x overhead

\item \verb+entropy_4k_fill+ \\
    To help the permutate function overcome the caching problem. Benchmarks
    shows 3x speedup. However, _4k_fill should be sure that the pages filling
    algorithm is using the same set of symbols for each page, otherwise the
    entropy of the whole file will increase more than the specified value. If
    you are using \verb+entropy_permutate_fill+ that will call cont_fill then comment the 
    symbols shuffle step. Still slower that _lookup_fill by a factor of 2.

\item \verb+entropy_lookup_fill+
    initializes a vector using _*_fill method then
    using that vector we will initialize our buffer by looking up
    different elements using a random index.The fastest method 
    after cont_fill it is slower by a factor of 3.
\end{itemize}



\section{Filebench Workflow Integration}
We used an \verb+#ifdef CONFIG_ENTROPY_DATA_EXPERIMENTAL+ construct to optionally include our code at make time. We primarily modified three of the existing files in filebench to integrate our entropy generation logic into the main workflow:
\subsection{fileset.h}
We added a \verb+struct source fs_ds+ as a member of \verb+struct fileset+. Although certain data is recorded by the parser into another member of \verb+struct fileset+ which is \verb+struct+ \verb+attr* fs_datasource+, we found it necessary to store this information again in \verb+struct+ \verb+source fs_ds+. The reasons for this were twofold. Firstly, all other attributes of a fileset are captured using the \verb+struct attr+ data structure. We decided not to modify this structure to remain compliant with the exising implementation. Secondly, it might be necessary to store derived results in the \verb+struct source fs_ds+. To store such derived results would modify the original \verb+struct attr* fs_datasource+ even more.

\subsection{fileset.c}
We defined an additional function \verb+fileset_init_datasource+ that will initialize the \verb+struct+ \verb+source fs_ds+ data structure depending on the information available in \verb+struct attr*+ \verb+fs_datasource+. A call to this function was added to two existing functions in this file:
\begin{enumerate}
\item \verb+fileset_create+
\item \verb+fileset_alloc_file+
\end{enumerate}
The function \verb+fileset_alloc_file+ performs an I/O operation to populate the created file with data. Initializing the \verb+struct+ \verb+source fs_ds+ member ensures that preallocated files are created with the desired entropy. Initializing the \verb+struct source fs_ds+ in \newline\verb+fileset_create+ ensures that all operations that will be done on files in this fileset will use the appropriate function to populate data in their buffers. This provides a generic way of populating data in files that is independent of the actual mechanism of populating data.

\subsection{flowop\_library.c}
We support the following filesets with entropy specifications:
\begin{enumerate}
\item createfiles
\item write
\item writewholefile
\item appendfile
\item appendfilerand
\end{enumerate}

\noindent These flow operations use the following functions in \verb+flowop_library.c+ to populate data in buffers:
\begin{enumerate}
\item \verb+flowoplib_write+
\item \verb+flowoplib_writewholefile+
\item \verb+flowoplib_appendfile+
\item \verb+flowoplib_appendfilerand+
\end{enumerate}

\noindent We inserted a call using the dynamically set function pointer to fill the buffer with the specified entropy value. An example of this is shown below:
\newline
\lstset{language=C}
\begin{lstlisting}
		int fd = flowop->fo_fdnumber;
		struct fileset *fs = threadflow->tf_fse[fd]->fse_fileset;
		fs->fs_ds.s_ops->fill(&fs->fs_ds, iobuf, iosize);
\end{lstlisting}

\noindent In the example shown above, the existing \verb+flowop+ and \verb+threadflow+ variables are used to get the corresponding fileset. Once a handle on the fileset is procured, its \verb+fill+ function is used to populate the preallocated buffer \verb+iobuf+ with the entropy value specified in \verb+fs_ds+.

\chapter{Setup}\label{chap:setup}


\paragraph{}
The simulations were run under the following conditions:
\begin{enumerate}
\item The machine specifications:
\begin{itemize}
\item 2 Intel Xeon 2.8 GHz, 2GB RAM
\item 2 73G SCSI Disks. The second disk was used for benchmarking.
\item Ubuntu 10.04 LTS Server Edition 64bit.
\item Software stack: JDK 7 Early Access Build 117, Fuse 2.8.1, SDFS 1.0.1
\end{itemize}
\item The base filesystem that SDFS read and write from is formatted as ext2 partition.

\item we mount the SDFS partition with default parameters. 
\item Between any two runs of Filebench, prepare.sh and cleanup.sh are called to the do the following:
    \begin{itemize}
    \item The SDFS partition is unmounted
    \item The chunkstore is cleared. 
    \item The ext2 partition is unmounted and formatted.
    \item Clear any filebench side effects stored at /var/tmp
    \item Drop memory caches by running \verb+sync && echo 3 > /proc/sys/vm/drop_caches+
    \end{itemize}

\item The write workload scenario is \\
\begin{lstlisting}
set $dir=/mnt/sdfs
define fileset name=rami_fileset,path=$dir,size=1m,entries=50000,dirwidth=100000,prealloc=0,datasource=entro,entropy=7.0
define process name=filewriter,instances=1
{
  thread name=filewriterthread,memsize=10m,instances=10
  {
    flowop createfile name=createfile,filesetname=rami_fileset,fd=1
    flowop write name=writtfile,fd=1,iosize=1m
    flowop closefile name=closefile,fd=1
  }
}
run 300
\end{lstlisting}

\item The read workload scenario is\\
\begin{lstlisting}
set $dir=/mnt/sdfs
define fileset name=rami_fileset,path=$dir,size=1m,entries=50000,dirwidth=100000,prealloc,datasource=entro,entropy=3.0
define process name=filereader,instances=1
{
  thread name=filereaderthread,memsize=10m,instances=10
  {
    flowop read name=readfile,filesetname=rami_fileset,iosize=1m
  }
}
run 300
\end{lstlisting}
\item The python script \verb+robot.py+, is a command line tool that reads the workload files and replace the entropy to the specified value in a file named values, then generates a temporary file that will be fed to Filebench. For example: \\
\lstset{language=bash}
\begin{lstlisting}
$./robot.py -v ./values -l results2 -d /mnt/sdfs/rami_fileset+ \
-x /mnt/sdbdrive/dedupfs/files/rami_fileset -w work_write.f
\end{lstlisting}
\end{enumerate}

\chapter{Results}\label{chap:res}
\chapter{Conclusions}\label{chap:conc}
\chapter{Future Work}\label{chap:fut}

%-----------------------------------------------------------
\addcontentsline{toc}{chapter}{\numberline{}Bibliography}
\bibliographystyle{plain}
\bibliography{bib}

%-----------------------------------------------------------
\end{document}
