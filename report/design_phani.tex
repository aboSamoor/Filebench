\chapter{Design}\label{chap:des}
\noindent The design of our project can be logically divided into three parts. Firstly, we need a mechanism to specify what we want Filebench to do, viz. to generate files with specific entropy values. Secondly, we need a way to generate data with arbitrary entropy. Lastly, we need a way to populate this data into files in the existing workflow of Filebench. The following sections describe and also explain the rationale behind the design of each of these parts.

\section{Filebench Interpreter}
We modified the Filebench Interpreter to enable Filebench to optionally accept various datasources. These datasources are meant to be used to populate files created during benchmarking process. The Interpreter has thus been modified to accept new attributes called \verb+datasource+ and \verb+entropy+ as a part of the fileset command. For example a valid fileset command is:\\

\indent \verb+define fileset name=bigfileset,...,datasource=entro,entropy=3.4+
\newline
\noindent The attribute \verb+entropy+ is not directly an attribute of \verb+fileset+ command. It is a subattribute of the datasource type. For example the below is an invalid fileset definition.\\

\verb+define fileset name=bigfileset,path=\$dir,...,entropy=3.4+\\
\newline
\noindent All attributes for command \verb+fileset+ have a place in the \verb+struct fileset+. The \verb+datasouce+ attribute has also been placed in the structure. A new place has been created in this structure for \verb+datasource+ attribute alone and not for \verb+entropy+ attribute. The logic is that, the only attribute that makes sense to be part of \verb+fileset+ is the type of data defined by datasource. The attributes which define the data itself of the \verb+datasource+ are irrelevant to be part of \verb+fileset+ datastructure.
\newline
\noindent Since the a datasource can have various attributes of the data constituting it, \verb+datasource+ object has a pointer to a list of attribute objects relevant to it. This is the reason \verb+datasource+ is of type \verb+attr+ and not \verb+avd_t+. To accommodate  a list of sub-attributes inside an attribute(like the \verb+datasource+), \verb+struct attr+ has been modified.
