\chapter{Setup}\label{chap:setup}


\paragraph{}
The simulations were run under the following conditions:
\begin{enumerate}
\item The machine specifications:
\begin{itemize}
\item 2 Intel Xeon 2.8 GHz, 2GB RAM
\item 2 73G SCSI Disks. The second disk was used for benchmarking.
\item Ubuntu 10.04 LTS Server Edition 64bit.
\item Software stack: JDK 7 Early Access Build 117, Fuse 2.8.1, SDFS 1.0.1
\end{itemize}
\item The base filesystem that SDFS read and write from is formatted as ext2 partition.

\item we mount the SDFS partition with default parameters. 
\item Between any two runs of Filebench, prepare.sh and cleanup.sh are called to the do the following:
    \begin{itemize}
    \item The SDFS partition is unmounted
    \item The chunkstore is cleared. 
    \item The ext2 partition is unmounted and formatted.
    \item Clear any filebench side effects stored at /var/tmp
    \item Drop memory caches by running \verb+sync && echo 3 > /proc/sys/vm/drop_caches+
    \end{itemize}

\item The write workload scenario is \\
\begin{lstlisting}
set $dir=/mnt/sdfs
define fileset name=rami_fileset,path=$dir,size=1m,entries=50000,dirwidth=100000,prealloc=0,datasource=entro,entropy=7.0
define process name=filewriter,instances=1
{
  thread name=filewriterthread,memsize=10m,instances=10
  {
    flowop createfile name=createfile,filesetname=rami_fileset,fd=1
    flowop write name=writtfile,fd=1,iosize=1m
    flowop closefile name=closefile,fd=1
  }
}
run 300
\end{lstlisting}

\item The read workload scenario is\\
\begin{lstlisting}
set $dir=/mnt/sdfs
define fileset name=rami_fileset,path=$dir,size=1m,entries=50000,dirwidth=100000,prealloc,datasource=entro,entropy=3.0
define process name=filereader,instances=1
{
  thread name=filereaderthread,memsize=10m,instances=10
  {
    flowop read name=readfile,filesetname=rami_fileset,iosize=1m
  }
}
run 300
\end{lstlisting}
\item The python script \verb+robot.py+, is a command line tool that reads the workload files and replace the entropy to the specified value in a file named values, then generates a temporary file that will be fed to Filebench. For example: \\
\lstset{language=bash}
\begin{lstlisting}
$./robot.py -v ./values -l results2 -d /mnt/sdfs/rami_fileset+ \
-x /mnt/sdbdrive/dedupfs/files/rami_fileset -w work_write.f
\end{lstlisting}
\end{enumerate}
